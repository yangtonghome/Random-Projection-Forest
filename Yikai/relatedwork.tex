\section{Related Work}
Random projection tree (RP tree) is widely used in the fields of machine learning and data mining. 
The projection method is first applied to map high-dimensional data randomly into low-dimensional space.
% A random projection from $d$ dimensions to $d'$ dimensions is a linear transformation represented by a $d\times d'$ matrix $R$, which is generated by first setting each entry of the matrix to a value drawn from a i.i.d N(0,1) distribution and then normalizing the columns to unit length. Given a $d$-dimensional data set represented as an $n\times d$ matrix $X$, where $n$ is the number of data points in $X$, the mapping $X\times R$ results in a reduced dimension data set $X'$
A random projection from $d-$dimensional sample space to $d'-$dimensional feature space can be expressed as a $d\times d'$ matrix $R$, where each column is a random unit vector. Given a $d-$dimensional data set containing $n$ samples and represented as an $n\times d$ matrix, the mapping $X\times R$ generates a low-dimensional sample data set $X'$.
The random projection tree is also used as an approximate replacement for the KD tree. Each node of the RP tree selects a random projection direction $d$, and computes the projection value $x\cdot d$ for all samples $x$ contained in the node. Select the value near the median of the projection value as the threshold to divide the samples. 

The most relevant work is the recently proposed random projection forest (RP forest).  The RP forest contains two different algorithms based on improvements to the two applications above. 
The first algorithm is an extension of the high-dimensional mapping. A random projection matrix is selected for each node of each tree in the forest for feature transformation, and an optimal feature is selected to divide the samples into two subtrees.
The second algorithm is an extension of the approximate KD tree, selecting a random projection direction and a random threshold at each node of each tree. Use thresholds to divide the samples into two substrees.

In the RP forest algorithm, we consider that the samples appearing in the same leaf node of the same tree are similar, and construct a similarity matrix according to this principle. The constructed similarity matrix can be regarded as a kernel matrix applied to the clustring algorithm.